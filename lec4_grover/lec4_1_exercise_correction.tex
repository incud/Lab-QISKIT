\begin{frame}[fragile]{Exercise: Quantum Adder}
\begin{minted}{python}
import numpy as np
from qiskit import *
from qiskit.circuit.library import QFT

np.set_printoptions(precision=4, suppress=True)

def decode_amplitudes(n, the_sv, prefix=''):
  for index, value in enumerate(the_sv):
    if np.abs(value) > 0.1:
      amplitude_name = format(index, "#0" + str(2 + 2 * n) + "b")
      print(f'{amplitude_name} has value {value:.3f}')
\end{minted}
\end{frame}

\begin{frame}[fragile]{Exercise: Quantum Adder}
\begin{minted}{python}
n = 2
qa = QuantumRegister(n, "a")
qb = QuantumRegister(n, "b")
cr = ClassicalRegister(n, "cr")
qc = QuantumCircuit(qa, qb, cr)
sv_sim = Aer.get_backend('statevector_simulator')

# encode states
qc.x(qa[0])
qc.x(qa[1])
qc.x(qb[0])
qc.x(qb[1])
\end{minted}
\end{frame}

\begin{frame}[fragile]{Exercise: Quantum Adder}
\begin{minted}{python}
# qft
qc.append(QFT(n), qa[:])

# adder
qc.cp(np.pi,   qb[0], qa[1])
qc.cp(np.pi/2, qb[0], qa[0])
qc.cp(np.pi,   qb[1], qa[0])

# iqft
qc.append(QFT(n, inverse=True), qa[:])

# print
sv = execute(qc, sv_sim).result().get_statevector()
decode_amplitudes(n, sv, prefix="after  IQFT:")
\end{minted}
\end{frame}
