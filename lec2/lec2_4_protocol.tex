\begin{frame}{The protocol (1)}
\begin{quantikz}[]
\lstick[wires=2]{Alice}\qw
    & \gate{init(\ket{\psi})}
    & \qw \\
    \qw
    & \qw
    & \qw \\
\lstick{Bob}\qw
    & \qw
    & \qw
\end{quantikz}

\bigskip Alice own the qubit in state \(\ket{\psi} = \alpha \ket{0} + \beta \ket{1}\)
\end{frame}

% =========================================

\begin{frame}{The protocol (2)}
    
\begin{quantikz}[]
\lstick[wires=2]{Alice}\qw
    & \gate{init(\ket{\psi})}
    & \qw \\
    \qw
    & \gate[wires=2]{bell}
    & \qw\\
\lstick{\(\ket{0}\)}\qw
    & \qw
    & \qw
\end{quantikz}

\bigskip Alice and Bob shares a Bell state \(\frac{1}{\sqrt{2}} (\ket{00} + \ket{11})\)
\end{frame}

% =========================================

\begin{frame}[fragile]{Composed gates}

Hierarchical organization of the circuit

\begin{minted}{python}
def create_bell_circuit():
	bell_circuit = QuantumCircuit(2, name='bell')
	bell_circuit.h(0)
	bell_circuit.cx(0, 1)
	return bell_circuit

psi = random_statevector(2).data
qr = QuantumRegister(3, 'q')
crx = ClassicalRegister(1, 'crx')
crz = ClassicalRegister(1, 'crz')
qc = QuantumCircuit(qr, crx, crz)
qc.initialize(psi, qr[0])
qc.append(create_bell_circuit(), [qr[0], qr[1]])
\end{minted}
\end{frame}

% =========================================

\begin{frame}{The protocol (3)}
    
\begin{quantikz}[]
\lstick[wires=2]{Alice}\qw
    & \gate{init(\ket{\psi})}
    & \ctrl{1} 
    & \gate{H} 
    & \qw \\
    \qw
    & \gate[wires=2]{bell}
    & \targ{}
    & \qw
    & \qw\\
\lstick{\(\ket{0}\)}\qw
    & \qw
    & \qw
    & \qw
    & \qw
\end{quantikz}

\bigskip Alice applies CNOT and H, the state becomes
\begin{align*}
    \frac{1}{2} \Big( & \ket{00}(\alpha \ket{0} + \beta \ket{1}) \\
                    + & \ket{01}(\alpha \ket{1} + \beta \ket{0}) \\
                    + & \ket{10}(\alpha \ket{0} - \beta \ket{1}) \\
                    + & \ket{10}(\alpha \ket{1} - \beta \ket{0})\Big) 
\end{align*}
\end{frame}

% =========================================

\begin{frame}{The protocol (4)}
    
\begin{quantikz}[]
\lstick[wires=2]{Alice}\qw
    & \gate{init(\ket{\psi})}
    & \ctrl{1} 
    & \gate{H} 
    & \meter{}
    & \cw
    & \cwbend{2}\\
    \qw
    & \gate[wires=2]{bell}
    & \targ{}
    & \qw
    & \meter{}
    & \cwbend{1} \\
\lstick{\(\ket{0}\)}\qw
    & \qw
    & \qw
    & \qw
    & \qw
    & \gate{X}
    & \gate{Z}
    & \qw
\end{quantikz}

\bigskip Alice measures and send the two classical bits, Bob adjusts its qubit according to the received bits:
\begin{align*}
    00 & \to I \\
    01 & \to X \\
    10 & \to Z \\
    11 & \to ZX
\end{align*}
\end{frame}

% =========================================

\begin{frame}[fragile]{The code (2)}
\begin{minted}{python}
psi = random_statevector(2).data

qr = QuantumRegister(3, 'q')
crx = ClassicalRegister(1, 'crx')
crz = ClassicalRegister(1, 'crz')
qc = QuantumCircuit(qr, crx, crz)
qc.initialize(psi, qr[0])
qc.append(create_bell_circuit(), [qr[1], qr[2]])

qc.cx(qr[0], qr[1])
qc.h(qr[0])
qc.measure(qr[0], crz)
qc.measure(qr[1], crx)
qc.x(qr[2]).c_if(crx, 1)
qc.z(qr[2]).c_if(crz, 1)
\end{minted}
\end{frame}

\begin{frame}[fragile]{The code (3)}
\begin{minted}{python}
sv_sim = Aer.get_backend('statevector_simulator')
sv = execute(qc, sv_sim).result().get_statevector()
\end{minted}

\bigskip Is \code{sv} the state \( \ket{\psi} \otimes \ket{b_2} \otimes \ket{b_1} \)?
\end{frame}

% =========================================

\begin{frame}[fragile]{The code (4)}

What is we use deferred measurements? \bigskip

\begin{minted}{python}
qr = QuantumRegister(3, 'q')
crx = ClassicalRegister(1, 'crx')
crz = ClassicalRegister(1, 'crz')
qc = QuantumCircuit(qr, crx, crz)
qc.initialize(psi, qr[0])
qc.append(create_bell_circuit(), [qr[1], qr[2]])

qc.cx(qr[0], qr[1])
qc.h(qr[0])
qc.cx(qr[0], qr[2])
qc.cz(qr[1], qr[2])
qc.measure(qr[0], crz)
qc.measure(qr[1], crx)
\end{minted}
\end{frame}

% =========================================

\begin{frame}{Running on real quantum computers}

\begin{itemize}
    \item Deferred measurement is mandatory;
    \item How can I check if the state is exactly the one Alice wanted to send?
\end{itemize}
\end{frame}

\begin{frame}{Running on real quantum computers}

\begin{itemize}
    \item Deferred measurement is mandatory;
    \item How can I check if the state is exactly the one Alice wanted to send?
\end{itemize}

\bigskip
\begin{center}
\begin{quantikz}[]
\lstick{\(\ket{0}\)}\qw
    & \gate{init}
    & \gate[wires=3]{teleport}
    & \qw \\
\lstick{\(\ket{0}\)}\qw
    & \qw
    & \qw
    & \qw \\
\lstick{\(\ket{0}\)}\qw
    & \qw
    & \qw
    & \gate{init^{-1}}
    & \qw\rstick{\(\ket{0}\)}
\end{quantikz}
\end{center}

\qquad Check if the last qubit is zero

\end{frame}

% =========================================

\begin{frame}[fragile]{Inverse of \code{initialize} gate}

\begin{minted}{python}
from qiskit.extensions import Initialize
psi = random_statevector(2).data

init_gate = Initialize(psi)
init_gate.label = 'init'

inverse_init_gate = init_gate.gates_to_uncompute()

qc = QuantumCircuit(1)
qc.append(init_gate, [0])
qc.append(inverse_init_gate, [0])
sv = execute(qc, sv_sim).result().get_statevector()
# sv is ALMOST ket zero
\end{minted}

\end{frame}

% =========================================

\begin{frame}[fragile]{Inverse of \code{initialize} gate}

Be careful: \code{initialize} contains a non-linear \code{Reset} gate \bigskip

\begin{minted}{python}
init_gate = Initialize(psi)
init_gate.inverse() # error

inverse_init_gate = init_gate.gates_to_uncompute() # no reset
init_wo_reset = inverse_init_gate.inverse()

init_wo_reset.inverse() # ok
\end{minted}

\end{frame}