%\begin{frame}[fragile]{Exercise: Quantum Adder}
\begin{minted}{python}
import numpy as np
from qiskit import *
from qiskit.circuit.library import QFT

np.set_printoptions(precision=4, suppress=True)

def decode_amplitudes(n, the_sv, prefix=''):
  for index, value in enumerate(the_sv):
    if np.abs(value) > 0.1:
      amplitude_name = format(index, "#0" + str(2 + 2 * n) + "b")
      print(f'{amplitude_name} has value {value:.3f}')
\end{minted}
\end{frame}

\begin{frame}[fragile]{Exercise: Quantum Adder}
\begin{minted}{python}
n = 2
qa = QuantumRegister(n, "a")
qb = QuantumRegister(n, "b")
cr = ClassicalRegister(n, "cr")
qc = QuantumCircuit(qa, qb, cr)
sv_sim = Aer.get_backend('statevector_simulator')

# encode states
qc.x(qa[0])
qc.x(qa[1])
qc.x(qb[0])
qc.x(qb[1])
\end{minted}
\end{frame}

\begin{frame}[fragile]{Exercise: Quantum Adder}
\begin{minted}{python}
# qft
qc.append(QFT(n), qa[:])

# adder
qc.cp(np.pi,   qb[0], qa[1])
qc.cp(np.pi/2, qb[0], qa[0])
qc.cp(np.pi,   qb[1], qa[0])

# iqft
qc.append(QFT(n, inverse=True), qa[:])

# print
sv = execute(qc, sv_sim).result().get_statevector()
decode_amplitudes(n, sv, prefix="after  IQFT:")
\end{minted}
\end{frame}


\section{Lecture 4:\\ Applications to\\ Machine Learning}
\SectionPage{}

\begin{frame}{Aspects of QC that can improve ML}

Several algorithms for quantum computers have been proposed which improve the performance of machine learning software. Here we consider some \emph{classification} problem. We assume that \(N\) is the size of the training set and \(M\) the number of features per sample. \bigskip

Samples $x \in \mathbb{R}^n$ are encoded through a piece of circuit called \emph{feature map} $U(x)$.

\end{frame}


\begin{frame}{Some approaches}
\begin{center}
\begin{tabular}{c c}
    Algorithm & Improvement \\ \midrule
    SWAP-test & Distance procedure from \(O(M)\) to \(O(\log(M))\)  \\[1em]
    QSVM & The quantum circuit act ``like" a kernel function  \\[1em]
    QNN & Sometimes lead to faster training  \\[1em]
\end{tabular}
\end{center}
\end{frame}

%\begin{frame}{Swap-test}

Efficient procedure to calculate the inner product. Since the circuit depth linear in the number of qubits, it can lead to an exponential speed-up (depending on how you encode the data).

\begin{center}
\begin{quantikz}
\lstick{$\ket{0}$}      & \gate{H} & \ctrl{2} & \gate{H} & \meter{} \\
\lstick{$\ket{\psi_1}$} & \qw      & \swap{1} & \qw      & \qw \\
\lstick{$\ket{\psi_2}$} & \qw      & \swap{0} & \qw      & \qw \\
\end{quantikz}
\end{center}

\[p(\text{measure }0) = \frac{1}{2} + \frac{1}{2}|\braket{\psi_1}{\psi_2}|^2\]

\only<2->{\bigskip \emph{Note}: \(\text{distance}(\psi_1, \psi_2)^2 = 2(1-\braket{\psi_1}{\psi_2}) \)}
\end{frame}

\begin{frame}{Swap-test math}
\begin{align*}
    & \ket{0}\ket{\psi_1}\ket{\psi_2} \\
    \to^{(1)}\quad
    & \frac{1}{\sqrt{2}}(\ket{0}+\ket{1})\ket{\psi_1}\ket{\psi_2} \\
    \to^{(2)}\quad & \frac{1}{\sqrt{2}}(\ket{0}\ket{\psi_1}\ket{\psi_2} + \ket{1}\ket{\psi_2}\ket{\psi_1}) \\
    \to^{(3)}\quad & \frac{1}{2}(\ket{0}\ket{\psi_1}\ket{\psi_2} + \ket{1}\ket{\psi_1}\ket{\psi_2} + \ket{0}\ket{\psi_2}\ket{\psi_1} - \ket{1}\ket{\psi_2}\ket{\psi_1}) \\
    = \quad & \frac{1}{2}
\ket{0}(\ket{\psi_1}\ket{\psi_2} + \ket{\psi_2}\ket{\psi_1}) + \frac{1}{2} \ket{1}(\ket{\psi_1}\ket{\psi_2} - \ket{\psi_2}\ket{\psi_1}) 
\end{align*}
\[ p(\text{measure zero}) = \frac{1}{2}(\ket{\psi_1}\ket{\psi_2} + \ket{\psi_2}\ket{\psi_1})^\dagger \frac{1}{2}(\ket{\psi_1}\ket{\psi_2} + \ket{\psi_2}\ket{\psi_1})\]
\end{frame}

\begin{frame}{C-SWAP with multiple qubits per state}
\begin{center}
    \begin{quantikz}
\lstick{$\ket{0}$}      & \gate{H} & \ctrl{2} & \ctrl{3} & \gate{H} & \meter{} \\
\lstick{$\ket{\psi}_0$} & \qw      & \swap{2} & \qw      & \qw      & \qw \\
\lstick{$\ket{\psi}_1$} & \qw      & \qw      & \swap{2} & \qw      & \qw \\
\lstick{$\ket{\phi}_0$} & \qw      & \swap{0} & \qw      & \qw      & \qw \\
\lstick{$\ket{\phi}_1$} & \qw      & \qw      & \swap{0} & \qw      & \qw \\
\end{quantikz}
\end{center}
\end{frame}

\begin{frame}[fragile]{SWAP-test on qiskit}
\begin{minted}{python}
n = 1 # qubit per register
psi = np.array([0, 1])
phi = np.array([1, 0])

qra, qr1, qr2 = QuantumRegister(1), QuantumRegister(n), QuantumRegister(n)
cr = ClassicalRegister(1)
qc = QuantumCircuit(qa, qr1, qr2, cr)
qc.h(qa[0])
qc.initialize(psi, qr1[:])
qc.initialize(phi, qr2[:])
for i in range(n):
    qc.cswap(qra[0], qr1[i], qr2[i])
qc.h(qra[0])
qc.measure(qra[0], cr[0])
\end{minted}
\end{frame}

\begin{frame}[fragile]{SWAP-test on qiskit}
\begin{minted}{python}
qasm_sim = Aer.get_backend('qasm_simulator')
SHOTS = 1000
count = execute(qc, qasm_sim, shots=SHOTS).result.get_count()
p0 = count.get('0', 0)/SHOTS
p1 = count.get('1', 0)/SHOTS
inner_prod = sqrt(2*max(p0 - 0.5, 0))
\end{minted}
\end{frame}


\begin{frame}{K-nearest neighbor classificator with swap-test}
\begin{enumerate}
    \item<1-> Consider some unlabelled instance \(\tilde{x}\) and some training set \((x_1, y_1); ...; (x_n, y_n)\) where \(y_i \in \{\ell_1, ..., \ell_m\}\)
    \item<2-> Calculate the inner product between \(\tilde{x}\) and any \(x_i\) through the swap-test;
    \item<3-> Pick the \(k\) elements whose inner product is bigger and their corresponding labels, infer \(\tilde{y}\) to be the most occurring label.
\end{enumerate}
\end{frame}


\begin{frame}[fragile]{Your turn!}
Build the swap-test circuit.

Build a classificator for IRIS problem, then calculate its accuracy. \bigskip

\begin{minted}{python}
from qiskit.ml.datasets import iris
sample_train, training_input, test_input, class_labels = 
    iris(training_size=40, test_size=10, n=4)
# len(training_input[key]) == 40 for key in ['A', 'B', 'C']
# len(test_input[key]) == 10 for key in ['A', 'B', 'C']

# fix qiskit.ml.datasets.iris code line 30: 
# test_size=test_size*len(class_labels)
\end{minted}
\end{frame}


\begin{frame}[fragile]{Your turn! (hint)}

\begin{minted}{python}
from heapq import nlargest
from operator import itemgetter
def most_common(lst): return max(set(lst), key=lst.count)

def quantum_classifier(x, training_set):
    k=3
    similarities = []
    for label in ['A', 'B', 'C']:
        for item in training_set[label]:
            the_ip = quantum_inner_product(x, item)
            similarities.append((the_ip, label))
    k_similarities = nlargest(k, similarities, key=itemgetter(0))
    _, k_labels = zip(*k_similarities)
    return most_common(k_labels)
\end{minted}
\end{frame}


\begin{frame}{More on swap-test classifier}

The Hadamard classifier (Schuld et al.) implements a whole classifier. 

\begin{center}
    \begin{quantikz}[]%
    \lstick{\(\ket{a}\)}\qw%
        & \gate{H}%
        & \ctrl{2}%
        & \gate{X}%
        & \ctrl{2}%
        & \qw%
        & \ctrl{2}%
        & \gate{H}%
        & \meter{discard 1} \\%
    \lstick{\(\ket{m}\)}\qw%
        & \gate{H}%
        & \qw%
        & \qw%
        & \ctrl{1}%
        & \gate{X}%
        & \ctrl{1} %
        & \ctrl{2}%
        & \qw\\%
    \lstick{\(\ket{d}\)}\qw%
        & \qw%
        & \gate{\tilde{x}}%
        & \qw%
        & \gate{x_1}%
        & \qw%
        & \gate{x_2}%
        & \qw%
        & \qw\\%
    \lstick{\(\ket{c}\)}\qw%
        & \qw%
        & \qw%
        & \qw%
        & \qw%
        & \qw%
        & \qw%
        & \targ{}%
        & \meter{}%
\end{quantikz}
\end{center}

\[p(\text{measure }\ket{c}\;0) = \begin{cases}
    <0.5, & d(\tilde{x}, x_1) < d(\tilde{x}, x_2) \\
    >0.5, & d(\tilde{x}, x_1) > d(\tilde{x}, x_2)
\end{cases} \]

\end{frame}

\begin{frame}[fragile]{Controlled-initialize}
\begin{minted}{python}
init_data = QuantumCircuit(n)
init_data.initialize(psi, range(n))
init_data_wo_reset = init_data.gates_to_uncompute().inverse()
control_init_data = init_data.control(1)

qc = ...
qc.append(control_init_data, ...)
\end{minted}
\end{frame}

\begin{frame}[fragile]{Your turn!}
Build the Hadamard classifier circuit. Use it to solve the IRIS problem. \bigskip

\end{frame}

%\begin{frame}{QSVM}
Sometimes data cannot be separated in the original feature space, but can be separated in another (higher dimensional) ones. 

\only<2->{\bigskip The embedding of data into a different space is called \emph{feature maps}. Once the data is embedded, the classification can be run into the higher dimensional space (NB: the only operation we care for is the inner product).}

\only<3->{\bigskip Each embedding of data into a quantum circuit act as a feature map, then the inner product is performed by the swap-test.}

\only<4->{\bigskip \emph{Difference with the previous classifier}: no performance improvement, potential accuracy improvement (depending on the efficacy of the feature map)}

\only<5->{\bigskip The procedure make sense only for feature map that cannot be simulated efficiently on classical hardware.}
\end{frame}


\begin{frame}[fragile]{Your turn!}
Try the QSVM. Check some feature map available on Qiskit documentation (find the meaning of \code{repetition} and \code{entanglement} parameters). Create a new feature map extending class \code{qiskit.aqua.components.feature\_maps.FeatureMap} (four qubit, one \code{RX} gate per each qubit - and feature of IRIS). \bigskip

\begin{minted}{python}
from qiskit.circuit.library import ZZFeatureMap
feature_map = ZZFeatureMap(feature_dimension=4, reps=1, entanglement='linear')
backend = Aer.get_backend('statevector_simulator')
quantum_instance = QuantumInstance(backend, shots=1))
qsvm = QSVM(feature_map, training_input, test_input)
result = qsvm.run(quantum_instance)
print(f'Testing success ratio: {result["testing_accuracy"]}')
\end{minted}
\end{frame}

\begin{frame}[fragile]{Parametrized circuits}
Circuit depending on some parameters.\bigskip

\begin{minted}{python}
theta = Parameter('theta')
qc = QuantumCircuit(1)
qc.rx(theta, 0)

# bind paramenters explicitly
qc2 = qc.bind_parameters({theta: 0.123})
qc2.draw()

# bind parameters when executing
execute(qc, backend=..., parameter_binds=[
    {theta: theta_val} for theta_val in [0.1, 0.2, 0.3])
\end{minted}
\end{frame}

\begin{frame}{Variational Quantum Classifier}
    
A variational quantum circuit contains parameterized gates whose values depend on some variables $\theta_1, ..., \theta_n$.

\begin{enumerate}
    \item Instantiate the circuit choosing random $\theta$s;
    \item Run the circuit and obtain output $y$;
    \item Use $y$ to compute a loss function and update $\theta$s accordingly.
\end{enumerate}

\bigskip\alert{Advantage}: The iterative optimization of the parameters allows us to circumvent the high-depth circuit.

\medskip\structure{Disadvantage}: No assurance of speedup. 
\end{frame}





\begin{frame}{Variational Quantum Classifier (2)}

\begin{center}
    \begin{quantikz}
    \lstick[wires=3]{$\ket{0}$} 
    & \gate[wires=3][2cm]{U(x)} & \gate[wires=3][2cm]{U(\theta)} & \meter{} \rstick[wires=3]{$y$} \\
    & \qw & \qw & \meter{}  \\
    & \qw & \qw & \meter{}  \\
    \end{quantikz}
\end{center}

\begin{itemize}
    \item input $x \in \mathbb{R}^n$ is encoded through \emph{feature map} $U(x)$;
    \item the state evolves through variational form $U(\theta)$;
    \item the parameters $\theta$s are chosen to minimize a certain loss function.
\end{itemize}

\end{frame}




\begin{frame}{Variational Quantum Classifier (3)}

Any variational circuit is a QNN. 
\begin{itemize}
    \item Mathematically, a QNN is a rather different object compared to classical NN;
    \item the analogy refers to:
    \begin{itemize}
        \item ``modular" nature of quantum gates in a circuit;
        \item use of tricks from training neural networks in the optimization of quantum algorithms.
    \end{itemize}
\end{itemize}

\begin{center}
    \scalebox{0.9}{
    \begin{quantikz}
        \qw
            & \gate{RY(\theta_1)} \gategroup[wires=3,steps=4,style={inner sep=-0.2pt}]{hidden layer 1}
            & \ctrl{1}           
            & \ctrl{2} 
            & \qw
            & \gate{RY(\theta_4)} \gategroup[wires=3,steps=4,style={inner sep=-0.2pt}]{hidden layer 2}
            & \ctrl{1}           
            & \ctrl{2} 
            & \qw 
            & \qw \\
        \qw 
            & \gate{RY(\theta_2)} 
            & \targ{}  
            & \qw      
            & \ctrl{1}
            & \gate{RY(\theta_5)} 
            & \targ{}  
            & \qw      
            & \ctrl{1}
            & \qw \\
        \qw 
            & \gate{RY(\theta_3)} 
            & \qw      
            & \targ{}  
            & \targ{}
            & \gate{RY(\theta_6)} 
            & \qw      
            & \targ{}  
            & \targ{}
            & \qw 
    \end{quantikz}}
\end{center}

\end{frame}


\begin{frame}{Barren plateau}

\begin{block}{Barren Plateau}
When minimizing the cost function, we might have exponentially vanishing gradients, known as barren plateau landscapes. In that case, the network cannot be efficiently trained.
\end{block}

\begin{itemize}
    \item Depends on the network architecture (Cerezo et al., 2020);
    \item Depends on the feature map (Abbas et al., 2020). 
\end{itemize}
\end{frame}


\begin{frame}[fragile]{Your turn!}
Try the VQC. \bigskip

\begin{minted}{python}
IRIS_FEATURES = 4
optimizer = ADAM(maxiter=100, lr=0.1)
feature_map = ZZFeatureMap(feature_dimension=IRIS_FEATURES)
var_form = TwoLocal(IRIS_FEATURES)
vqc = VQC(optimizer, feature_map, var_form, 
    training_input, test_input)

backend = Aer.get_backend('statevector_simulator')
quantum_instance = QuantumInstance(backend, shots=1)
result = vqc.run(quantum_instance)
\end{minted}
\end{frame}
