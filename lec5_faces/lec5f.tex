\section{Lecture 5:\\ Variational circuits}
\SectionPage{}

\input{lec5_ml/lec5_3_variational_classifier}

\section{Lecture 5:\\ Hadamard classifier}
\SectionPage{}

%\begin{frame}{The problem}
%Facial Expression Recognition (FER) is an extremely relevant task associated with human-computer interaction, with applications in predictive environments, content analysis, support for healthcare, and many more. 

%\bigskip FER is a classification problem, in which each picture representing a face have to be associated with its expression.

%\bigskip It is possible to solve FER with supervised learning approaches starting from a dataset for labelled pictures.
%\end{frame}



%\begin{frame}{The problem (2)}
%FER is solved using a quantum circuit that encodes:
%\begin{itemize}
%    \item a subset of the dataset;
%    \item the unlabelled test instance
%\end{itemize} 
%into quantum states.

%\bigskip Then, infer the test label as the one of the closest item in terms of Euclidian distance.
%\end{frame}


%\begin{frame}{The dataset}

The Extended Cohn-Kanade contains pictures from 123 different people from 18 to 50 years.

\bigskip \includegraphics[width=.48\textwidth]{img/lec5f/S010_004_00000019.png}%
\includegraphics[width=.48\textwidth]{img/lec5f/S010_006_00000015.png}

\bigskip Each picture is labelled with one of the following emotion: anger, contempt, disgust, fear, happiness, sadness, and surprise.
    
\end{frame}
%\begin{frame}{Active Appearance Models}

Each pictures is processed with Active Appearance Models.

\bigskip Consider a cloud of 68 points representing some abstract face, in which each point (\emph{landmark point}) is some particular point of the face.

\bigskip\centering\includegraphics[width=.5\textwidth]{img/lec5f/template.jpg}
\end{frame}

\begin{frame}{Active Appearance Models (2)}
The cloud of points is deformed in order to stick with the face in the picture.

\bigskip\centering\includegraphics[width=.6\textwidth]{img/lec5f/S046_005_00000023.png}
\end{frame}
%\begin{frame}{Faces \(\to\) graphs}

We keep only point related to the mouth. 

\bigskip In order to obtain a \textit{weighted graph} from the previous data cloud we opted for two different strategies:

\only<1>{

\begin{itemize}
	\item \textbf{Complete graph} whose vertices are the landmark points of the mouth and  edge-weights $ w_{ij} $ are equal to  the Euclidean distance between vertices $ i $ and $ j $.
\end{itemize}	

\bigskip\centering\includegraphics[width=0.5\textwidth]{img/lec5f/boccacomplete.png}}

\only<2>{

\begin{itemize}
	\item \textbf{Meshed graph} obtained using  the Delaunay triangulation algorithm of the  mouth  landmark points (complexity $ O(n \log n) $), also in this case the   edge-weights $ w_{ij} $ are equal to  the Euclidean distance between vertices $ i $ and $ j $.
\end{itemize}	

\bigskip\centering\includegraphics[width=0.5\textwidth]{img/lec5f/boccamesh.png}}

\end{frame}

\begin{frame}{Graph representation}

The graph is represented through its adjacency matrix.

\bigskip We keep only the upper triangular without diagonal, since the graph is undirected (symmetric matrix) and without self loop (the distance between any point with itself is zero). 

\bigskip Vectorize the matrix.

\end{frame}
\begin{frame}{Classical classifier}

The classifier receives in input the vector $x_\text{test}$ and two vectors $x_0, x_1$ which are one representative of the happy and sad faces.

\bigskip The Euclidian distances $\mathrm{distance}(x_\text{test}, x_0)$ and $\mathrm{distance}(x_\text{test}, x_1)$ are calculated. The test instance is classifies as the closest representative, whose distance is lower. 

\end{frame}

\begin{frame}[fragile]{The code}

\begin{minted}{python}
def classical_distance(G_0, G_1, G_test, tol=0.00001):
    
    # distance between the test instance and the happy instance
    distance_0 = np.linalg.norm(G_0 - G_test)
    # distance between the test instance and the sad instance
    distance_1 = np.linalg.norm(G_1 - G_test)
    # the difference is = 0 if the distance is equal, > 0 if sad is closest, < 0 if happy is closest
    difference = distance_0 - distance_1
    # remove some numerical error that can classify numbers like +-0.00001 as Y1 or Y0 instead of EQUALS
    the_difference = 0 if np.abs(difference) <= tol else difference
    
    # sign(difference) = 0 -> EQUAL; sign(difference) = 1 -> Y1; sign(difference) = -1 -> Y0
    label = int(np.sign(the_difference))
    return difference, ["EQUAL", "Y1", "Y0"][label]
\end{minted}
\end{frame}



\begin{frame}{Quantum classifier}
    \begin{center}
        \scalebox{0.8}{\begin{quantikz}[]
    \lstick{\(\ket{0}_a\)}\qw
        & \gate{H}
        & \ctrl{2}
        & \gate{X}
        & \ctrl{2}
        & \qw
        & \ctrl{2}
        & \gate{H}
        & \meter{\text{discard }1}
        & \cwbend{3}
        & \cw \rstick{a} \\
    \lstick{\(\ket{0}_i\)}\qw
        & \gate{H}
        & \qw
        & \qw
        & \ctrl{1}
        & \gate{X}
        & \ctrl{1} 
        & \ctrl{2}
        & \qw
        & \qw & \qw \\
    \lstick{\(\ket{0\cdots 00}_g\)}\qw
        & \qw
        & \gate{\mathbf{G_\text{\textbf test}}}
        & \qw
        & \gate{\mathbf{G^{\{1\}}}}
        & \qw
        & \gate{\mathbf{G^{\{2\}}}}
        & \qw
        & \qw
        & \qw & \qw \\
    \lstick{\(\ket{0}_c\)}\qw
        & \qw
        & \qw
        & \qw
        & \qw
        & \qw
        & \qw
        & \targ{}
        & \qw
        & \meter{}
        & \cw\rstick{c}
\end{quantikz}}
    \end{center}
\end{frame}

\begin{frame}{Quantum classifier}
The circuit evolves as it follows:
\only<1>{\begin{itemize}
\item the circuit starts in state:
$$|0\rangle_a |0\rangle_i |0\rangle_d |0\rangle_c;$$
\item after the first two Hadamard gates the state is:
$$\frac{1}{2} (|0\rangle+|1\rangle)_a (|0\rangle+|1\rangle)_i |0\rangle_d |0\rangle_c;$$
\end{itemize}
}%
%
\only<2>{\begin{itemize}
\item after the controlled initialization of the test instance the state is: 
$$\frac{1}{2} |0\rangle_a (|0\rangle+|1\rangle)_i |0\rangle_d |0\rangle_c
+ \frac{1}{2} |1\rangle_a (|0\rangle+|1\rangle)_i |G_\text{test}\rangle_d |0\rangle_c;$$
\item after the X operation on the ancilla qubit the state is:
$$\frac{1}{2} |0\rangle_a (|0\rangle+|1\rangle)_i |G_\text{test}\rangle_d |0\rangle_c
+ \frac{1}{2} |1\rangle_a (|0\rangle+|1\rangle)_i |0\rangle_d |0\rangle_c;$$
\end{itemize}
}%
%
\only<3>{\begin{itemize}
\item after the double-controlled initialization of the first representative the state is:
$$\frac{1}{2} |0\rangle_a (|0\rangle+|1\rangle)_i |G_\text{test}\rangle_d |0\rangle_c
+ \frac{1}{2} |1\rangle_a |0\rangle_i |0\rangle_d |0\rangle_c
+ \frac{1}{2} |1\rangle_a |1\rangle_i |G_0\rangle_d |0\rangle_c;$$
\item after the X operation on the index register the state is:
$$\frac{1}{2} |0\rangle_a (|0\rangle+|1\rangle)_i |G_\text{test}\rangle_d |0\rangle_c
+ \frac{1}{2} |1\rangle_a |0\rangle_i |G_0\rangle_d |0\rangle_c
+ \frac{1}{2} |1\rangle_a |1\rangle_i |0\rangle_d |0\rangle_c;$$
\item after the double-controlled initialization of the second representative the state is:
$$\frac{1}{2} |0\rangle_a (|0\rangle+|1\rangle)_i |G_\text{test}\rangle_d |0\rangle_c
+ \frac{1}{2} |1\rangle_a |0\rangle_i |G_0\rangle_d |0\rangle_c
+ \frac{1}{2} |1\rangle_a |1\rangle_i |G_1\rangle_d |0\rangle_c;$$
\end{itemize}
}%
%
\only<4>{\begin{itemize}
\item after the CNOT gate binding the index register with the class register, the state is:
{\footnotesize $$\frac{1}{2} |0\rangle_a |0\rangle_i |G_\text{test}\rangle_d |0\rangle_c
+ \frac{1}{2} |0\rangle_a |1\rangle_i |G_\text{test}\rangle_d |1\rangle_c
+ \frac{1}{2} |1\rangle_a |0\rangle_i |G_0\rangle_d |0\rangle_c
+ \frac{1}{2} |1\rangle_a |1\rangle_i |G_1\rangle_d |1\rangle_c;$$}%

for the sake of visualization, we set $|0\rangle_c$ as $|y_0\rangle_c$ and $|1\rangle_c$ as $|y_1\rangle_c$ to remind us that such register contains the two labels:
{\footnotesize $$\frac{1}{2} |0\rangle_a |0\rangle_i |G_\text{test}\rangle_d |y_0\rangle_c
+ \frac{1}{2} |0\rangle_a |1\rangle_i |G_\text{test}\rangle_d |y_1\rangle_c
+ \frac{1}{2} |1\rangle_a |0\rangle_i |G_0\rangle_d |y_0\rangle_c
+ \frac{1}{2} |1\rangle_a |1\rangle_i |G_1\rangle_d |y_1\rangle_c;$$}
which is re-arranged into the following equation:
$$ \frac{1}{2} \sum_{k \in \{0, 1\}} \Big( |0\rangle_a |G_\text{test}\rangle_d + |1\rangle_a |G_k\rangle_d  \Big) |k\rangle_i |y_k\rangle_c;$$
\end{itemize}
}%
%
\only<5>{\begin{itemize}
\item after the final Hadamard gate the state is: $$\frac{1}{2\sqrt{2}} \sum_{k \in \{0, 1\}} \Big( |0\rangle_a (|G_\text{test}\rangle + |G_k\rangle)_d + |1\rangle_a (|G_\text{test}\rangle - |G_k\rangle)_d \Big) |k\rangle_i |y_k\rangle_c.$$
\end{itemize}}
\end{frame}

\begin{frame}{Measurement interpretation}

By estimating the probability of reading class $y_0=0$ or $y_1=1$ we can estimate the distance between $G_\text{test}, G_0$ with respect to the distance between $G_\text{test}, G_1$. So, 

$$y_\text{test} = \begin{cases}
    y_0,           & > .5 \\
    y_1,           & < .5 \\
    \text{equals}, & = .5
\end{cases}$$

To mitigate the errors due to the probabilistic nature of the computation, we can introduce a small tollerance $\epsilon$ around the boundary between the two labels:

$$y_\text{test} = \begin{cases}
    y_0, & > .5+\epsilon \\
    y_1, & < .5-\epsilon \\
    \text{equals}, & \text{otherwise}
\end{cases}$$
\end{frame}
\begin{frame}[fragile]{Your turn!}
\begin{minted}{python}
def quantum_distance_circuit(G_0, G_1, G_test):

    # calculate how many qubits are needed for the data register
    qubit_per_feature = int(np.ceil(np.log2(len(G_0))))
    
    # zero padding and normalization
    x_0 = normalize(zero_padding(G_0, qubit_per_feature))
    x_1 = normalize(zero_padding(G_1, qubit_per_feature))
    x_test = normalize(zero_padding(G_test, qubit_per_feature))
    
    # define the registers
    qr_ancilla = QuantumRegister(1, 'anc')
    qr_index = QuantumRegister(1, 'index')
    qr_data = QuantumRegister(qubit_per_feature, 'data')
    qr_class = QuantumRegister(1, 'class')
    cr_ancilla = ClassicalRegister(1, 'cr_anc')
    cr_class = ClassicalRegister(1, 'cr_class')
\end{minted}
\end{frame}

\begin{frame}[fragile]{Your turn!}
\begin{minted}{python}
    # initialize the circuit
    qc = QuantumCircuit(qr_ancilla, qr_index, qr_data, qr_class, cr_ancilla, cr_class)
    
    # prepare initialization circuits
    # ...
    
    # initialize index and data registers
    # ...
    
    # correlate the index with the class
    # ...
\end{minted}
\end{frame}

\begin{frame}[fragile]{Your turn!}
\begin{minted}{python}
    # work on ancilla
    # ...
    # measure
    qc.measure(qr_ancilla[0], cr_ancilla[0])
    qc.measure(qr_class[0], cr_class[0])
    return qc
\end{minted}
\end{frame}

\begin{frame}[fragile]{Your turn!}
\begin{minted}{python}
def quantum_distance(G_0, G_1, G_test, tol=0.00001):
    SHOTS = 10000
    qc = quantum_distance_circuit(G_0, G_1, G_test)
    counts = execute(qc, Aer.get_backend('qasm_simulator'), shots=SHOTS).result().get_count()
    
    # add missing values
    keys = ['0 0', '0 1', '1 0', '1 1'] # LSB = Ancilla, MSB = Class
    for key in keys:
        if key not in counts:
            counts[key] = 0

    # calculate distance
    # ...
    difference = (distance_happy - distance_sad) / SHOTS
    the_difference = 0 if np.abs(difference) <= tol else difference
    return difference, ["EQUAL", "Y1", "Y0"][int(np.sign(the_difference))]
\end{minted}
\end{frame}

\begin{frame}[fragile]{Test the correctness}
\begin{minted}{python}
def test_accuracy_of_classification(states):
    correct, wrong = 0, 0
    for i, first_data in enumerate(states):
        for j, second_data in enumerate(states):
            for k, third_data in enumerate(states):
                dc, cl = classical_distance(first_data,
                    second_data, third_data)
                dq, ql = quantum_distance(first_data,
                    second_data, third_data, tol=0.015)
                if cl == ql:
                    correct += 1
                    print(".", end="")
                else:
                    wrong += 1
                    print("X", end="")
    return (correct, wrong)
\end{minted}
\end{frame}

\begin{frame}[fragile]{Test the correctness}
\begin{minted}{python}
# create some quantum states
zero  = np.array([1, 0])
one   = np.array([0, 1])
plus  = (1/np.sqrt(2)) * np.array([1, 1])
minus = (1/np.sqrt(2)) * np.array([1, -1])
ipos  = (1/np.sqrt(2)) * np.array([1, complex(0,1)])
ineg  = (1/np.sqrt(2)) * np.array([1, complex(0,-1)])

# run the test
states = [zero, one, plus, minus, ipos, ineg]
correct, wrong = test_accuracy_of_classification(states)

# print results
print(f"\nClassified {correct} correctly and {wrong} wrongly")
\end{minted}
\end{frame}

\begin{frame}{Your turn!}
Estimate the accuracy of both classical and quantum classifier with IRIS.
\end{frame}