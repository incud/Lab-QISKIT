\begin{frame}{Classical classifier}

The classifier receives in input the vector $x_\text{test}$ and two vectors $x_0, x_1$ which are one representative of the happy and sad faces.

\bigskip The Euclidian distances $\mathrm{distance}(x_\text{test}, x_0)$ and $\mathrm{distance}(x_\text{test}, x_1)$ are calculated. The test instance is classifies as the closest representative, whose distance is lower. 

\end{frame}

\begin{frame}[fragile]{The code}

\begin{minted}{python}
def classical_distance(G_0, G_1, G_test, tol=0.00001):
    
    # distance between the test instance and the happy instance
    distance_0 = np.linalg.norm(G_0 - G_test)
    # distance between the test instance and the sad instance
    distance_1 = np.linalg.norm(G_1 - G_test)
    # the difference is = 0 if the distance is equal, > 0 if sad is closest, < 0 if happy is closest
    difference = distance_0 - distance_1
    # remove some numerical error that can classify numbers like +-0.00001 as Y1 or Y0 instead of EQUALS
    the_difference = 0 if np.abs(difference) <= tol else difference
    
    # sign(difference) = 0 -> EQUAL; sign(difference) = 1 -> Y1; sign(difference) = -1 -> Y0
    label = int(np.sign(the_difference))
    return difference, ["EQUAL", "Y1", "Y0"][label]
\end{minted}
\end{frame}



\begin{frame}{Quantum classifier}
    \begin{center}
        \scalebox{0.8}{\begin{quantikz}[]
    \lstick{\(\ket{0}_a\)}\qw
        & \gate{H}
        & \ctrl{2}
        & \gate{X}
        & \ctrl{2}
        & \qw
        & \ctrl{2}
        & \gate{H}
        & \meter{\text{discard }1}
        & \cwbend{3}
        & \cw \rstick{a} \\
    \lstick{\(\ket{0}_i\)}\qw
        & \gate{H}
        & \qw
        & \qw
        & \ctrl{1}
        & \gate{X}
        & \ctrl{1} 
        & \ctrl{2}
        & \qw
        & \qw & \qw \\
    \lstick{\(\ket{0\cdots 00}_g\)}\qw
        & \qw
        & \gate{\mathbf{G_\text{\textbf test}}}
        & \qw
        & \gate{\mathbf{G^{\{1\}}}}
        & \qw
        & \gate{\mathbf{G^{\{2\}}}}
        & \qw
        & \qw
        & \qw & \qw \\
    \lstick{\(\ket{0}_c\)}\qw
        & \qw
        & \qw
        & \qw
        & \qw
        & \qw
        & \qw
        & \targ{}
        & \qw
        & \meter{}
        & \cw\rstick{c}
\end{quantikz}}
    \end{center}
\end{frame}

\begin{frame}{Quantum classifier}
The circuit evolves as it follows:
\only<1>{\begin{itemize}
\item the circuit starts in state:
$$|0\rangle_a |0\rangle_i |0\rangle_d |0\rangle_c;$$
\item after the first two Hadamard gates the state is:
$$\frac{1}{2} (|0\rangle+|1\rangle)_a (|0\rangle+|1\rangle)_i |0\rangle_d |0\rangle_c;$$
\end{itemize}
}%
%
\only<2>{\begin{itemize}
\item after the controlled initialization of the test instance the state is: 
$$\frac{1}{2} |0\rangle_a (|0\rangle+|1\rangle)_i |0\rangle_d |0\rangle_c
+ \frac{1}{2} |1\rangle_a (|0\rangle+|1\rangle)_i |G_\text{test}\rangle_d |0\rangle_c;$$
\item after the X operation on the ancilla qubit the state is:
$$\frac{1}{2} |0\rangle_a (|0\rangle+|1\rangle)_i |G_\text{test}\rangle_d |0\rangle_c
+ \frac{1}{2} |1\rangle_a (|0\rangle+|1\rangle)_i |0\rangle_d |0\rangle_c;$$
\end{itemize}
}%
%
\only<3>{\begin{itemize}
\item after the double-controlled initialization of the first representative the state is:
$$\frac{1}{2} |0\rangle_a (|0\rangle+|1\rangle)_i |G_\text{test}\rangle_d |0\rangle_c
+ \frac{1}{2} |1\rangle_a |0\rangle_i |0\rangle_d |0\rangle_c
+ \frac{1}{2} |1\rangle_a |1\rangle_i |G_0\rangle_d |0\rangle_c;$$
\item after the X operation on the index register the state is:
$$\frac{1}{2} |0\rangle_a (|0\rangle+|1\rangle)_i |G_\text{test}\rangle_d |0\rangle_c
+ \frac{1}{2} |1\rangle_a |0\rangle_i |G_0\rangle_d |0\rangle_c
+ \frac{1}{2} |1\rangle_a |1\rangle_i |0\rangle_d |0\rangle_c;$$
\item after the double-controlled initialization of the second representative the state is:
$$\frac{1}{2} |0\rangle_a (|0\rangle+|1\rangle)_i |G_\text{test}\rangle_d |0\rangle_c
+ \frac{1}{2} |1\rangle_a |0\rangle_i |G_0\rangle_d |0\rangle_c
+ \frac{1}{2} |1\rangle_a |1\rangle_i |G_1\rangle_d |0\rangle_c;$$
\end{itemize}
}%
%
\only<4>{\begin{itemize}
\item after the CNOT gate binding the index register with the class register, the state is:
{\footnotesize $$\frac{1}{2} |0\rangle_a |0\rangle_i |G_\text{test}\rangle_d |0\rangle_c
+ \frac{1}{2} |0\rangle_a |1\rangle_i |G_\text{test}\rangle_d |1\rangle_c
+ \frac{1}{2} |1\rangle_a |0\rangle_i |G_0\rangle_d |0\rangle_c
+ \frac{1}{2} |1\rangle_a |1\rangle_i |G_1\rangle_d |1\rangle_c;$$}%

for the sake of visualization, we set $|0\rangle_c$ as $|y_0\rangle_c$ and $|1\rangle_c$ as $|y_1\rangle_c$ to remind us that such register contains the two labels:
{\footnotesize $$\frac{1}{2} |0\rangle_a |0\rangle_i |G_\text{test}\rangle_d |y_0\rangle_c
+ \frac{1}{2} |0\rangle_a |1\rangle_i |G_\text{test}\rangle_d |y_1\rangle_c
+ \frac{1}{2} |1\rangle_a |0\rangle_i |G_0\rangle_d |y_0\rangle_c
+ \frac{1}{2} |1\rangle_a |1\rangle_i |G_1\rangle_d |y_1\rangle_c;$$}
which is re-arranged into the following equation:
$$ \frac{1}{2} \sum_{k \in \{0, 1\}} \Big( |0\rangle_a |G_\text{test}\rangle_d + |1\rangle_a |G_k\rangle_d  \Big) |k\rangle_i |y_k\rangle_c;$$
\end{itemize}
}%
%
\only<5>{\begin{itemize}
\item after the final Hadamard gate the state is: $$\frac{1}{2\sqrt{2}} \sum_{k \in \{0, 1\}} \Big( |0\rangle_a (|G_\text{test}\rangle + |G_k\rangle)_d + |1\rangle_a (|G_\text{test}\rangle - |G_k\rangle)_d \Big) |k\rangle_i |y_k\rangle_c.$$
\end{itemize}}
\end{frame}

\begin{frame}{Measurement interpretation}

By estimating the probability of reading class $y_0=0$ or $y_1=1$ we can estimate the distance between $G_\text{test}, G_0$ with respect to the distance between $G_\text{test}, G_1$. So, 

$$y_\text{test} = \begin{cases}
    y_0,           & > .5 \\
    y_1,           & < .5 \\
    \text{equals}, & = .5
\end{cases}$$

To mitigate the errors due to the probabilistic nature of the computation, we can introduce a small tollerance $\epsilon$ around the boundary between the two labels:

$$y_\text{test} = \begin{cases}
    y_0, & > .5+\epsilon \\
    y_1, & < .5-\epsilon \\
    \text{equals}, & \text{otherwise}
\end{cases}$$
\end{frame}