\begin{frame}[fragile]{Your turn!}
\begin{minted}{python}
def quantum_distance_circuit(G_0, G_1, G_test):

    # calculate how many qubits are needed for the data register
    qubit_per_feature = int(np.ceil(np.log2(len(G_0))))
    
    # zero padding and normalization
    x_0 = normalize(zero_padding(G_0, qubit_per_feature))
    x_1 = normalize(zero_padding(G_1, qubit_per_feature))
    x_test = normalize(zero_padding(G_test, qubit_per_feature))
    
    # define the registers
    qr_ancilla = QuantumRegister(1, 'anc')
    qr_index = QuantumRegister(1, 'index')
    qr_data = QuantumRegister(qubit_per_feature, 'data')
    qr_class = QuantumRegister(1, 'class')
    cr_ancilla = ClassicalRegister(1, 'cr_anc')
    cr_class = ClassicalRegister(1, 'cr_class')
\end{minted}
\end{frame}

\begin{frame}[fragile]{Your turn!}
\begin{minted}{python}
    # initialize the circuit
    qc = QuantumCircuit(qr_ancilla, qr_index, qr_data, qr_class, cr_ancilla, cr_class)
    
    # prepare initialization circuits
    # ...
    
    # initialize index and data registers
    # ...
    
    # correlate the index with the class
    # ...
\end{minted}
\end{frame}

\begin{frame}[fragile]{Your turn!}
\begin{minted}{python}
    # work on ancilla
    # ...
    # measure
    qc.measure(qr_ancilla[0], cr_ancilla[0])
    qc.measure(qr_class[0], cr_class[0])
    return qc
\end{minted}
\end{frame}

\begin{frame}[fragile]{Your turn!}
\begin{minted}{python}
def quantum_distance(G_0, G_1, G_test, tol=0.00001):
    SHOTS = 10000
    qc = quantum_distance_circuit(G_0, G_1, G_test)
    counts = execute(qc, Aer.get_backend('qasm_simulator'), shots=SHOTS).result().get_count()
    
    # add missing values
    keys = ['0 0', '0 1', '1 0', '1 1'] # LSB = Ancilla, MSB = Class
    for key in keys:
        if key not in counts:
            counts[key] = 0

    # calculate distance
    # ...
    difference = (distance_happy - distance_sad) / SHOTS
    the_difference = 0 if np.abs(difference) <= tol else difference
    return difference, ["EQUAL", "Y1", "Y0"][int(np.sign(the_difference))]
\end{minted}
\end{frame}

\begin{frame}[fragile]{Test the correctness}
\begin{minted}{python}
def test_accuracy_of_classification(states):
    correct, wrong = 0, 0
    for i, first_data in enumerate(states):
        for j, second_data in enumerate(states):
            for k, third_data in enumerate(states):
                dc, cl = classical_distance(first_data,
                    second_data, third_data)
                dq, ql = quantum_distance(first_data,
                    second_data, third_data, tol=0.015)
                if cl == ql:
                    correct += 1
                    print(".", end="")
                else:
                    wrong += 1
                    print("X", end="")
    return (correct, wrong)
\end{minted}
\end{frame}

\begin{frame}[fragile]{Test the correctness}
\begin{minted}{python}
# create some quantum states
zero  = np.array([1, 0])
one   = np.array([0, 1])
plus  = (1/np.sqrt(2)) * np.array([1, 1])
minus = (1/np.sqrt(2)) * np.array([1, -1])
ipos  = (1/np.sqrt(2)) * np.array([1, complex(0,1)])
ineg  = (1/np.sqrt(2)) * np.array([1, complex(0,-1)])

# run the test
states = [zero, one, plus, minus, ipos, ineg]
correct, wrong = test_accuracy_of_classification(states)

# print results
print(f"\nClassified {correct} correctly and {wrong} wrongly")
\end{minted}
\end{frame}

\begin{frame}{Your turn!}
Estimate the accuracy of both classical and quantum classifier with IRIS.
\end{frame}