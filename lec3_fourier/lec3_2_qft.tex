
\begin{frame}{QFT}
Such matrix is \alert{unitary} thus can be implemented with a quantum circuit. 

\begin{align*}
    F^\dagger F 
    & = 
    \left(\frac{1}{\sqrt{N}} \sum_{j, k=0}^{N-1} \omega^{-jk} \ketbra{j}{k} \right)
    \left(\frac{1}{\sqrt{N}} \sum_{j', k'=0}^{N-1} \omega^{j'k'} \ketbra{k'}{j'} \right) \\
    & = 
    \frac{1}{N}
    \sum_{j, k, j', k'=0}^{N-1} \omega^{(j'k'-jk)} \ketbra{j}{k}\ketbra{k'}{j'} \text{ then } \braket{k}{k'} = \delta_{k, k'}\\
    & = 
    \frac{1}{N}
    \sum_{j, k, j'=0}^{N-1} \omega^{(j'-j)k} \ketbra{j}{j'} \text{ then } \sum_{k=0}^{N-1} \omega^{(j'-j)k} = \delta_{j, j'} \\
    & = 
    \frac{1}{N}
    \sum_{j,j'=0}^{N-1} \delta_{j, j'} \ketbra{j}{j'} = \mathbb{I}
\end{align*}
\end{frame}

% ===================================

\begin{frame}{QFT}
The circuit is called \alert{QFT} and can be implemented \alert{VERY} efficiently. The number of gates is between \((\log N)^2\) and \((\log N)^3\).
\end{frame}

% ===================================

\begin{frame}[fragile]{QFT}
Check the QFT matrix and DFT are equals.

\bigskip
\begin{minted}{python}
import numpy as np
from scipy.linalg import dft
from qiskit import *
from qiskit.circuit.library import QFT

n = 2 # number of qubits
dft_matrix = (1/np.sqrt(2**n)) * dft(2**n).conj()
u_sim = Aer.get_backend('unitary_simulator')
qft_matrix = execute(QFT(n), u_sim).result().get_unitary()

np.allclose(dft_matrix, qft_matrix) # true
\end{minted}
\end{frame}

% ===================================

%\begin{frame}{DTF example}
%\vspace{-1cm}
%\begin{align*}
%    \onslide<1->{\mqty[x_0 \\ x_1] = \mqty[1 \\ 2] 
%    \xrightarrow{\text{DFT}}
%    \mqty[y_0 \\ y_1] & =
%    \frac{1}{\sqrt{2}}
%    \mqty[\left(
%            x_0 \, e^{\frac{2 \pi i}{2} \cdot 0 \cdot 0}
%            +
%            x_1 \, e^{\frac{2 \pi i}{2} \cdot 0 \cdot 1}
%        \right)  \\ 
%        \left(
%            x_0 \, e^{\frac{2 \pi i}{2} \cdot 0 \cdot 1}
%            +
%            x_1 \, e^{\frac{2 \pi i}{2} \cdot 1 \cdot 1}
%        \right) 
%    ]
%    & & =
%    \frac{1}{\sqrt{2}} \mqty[3 \\ -1] \\[1em]}
%    %
%    \onslide<2->{
%    \left[\omega = e^{\frac{2 \pi i}{N}}\right] \;\;\qquad\qquad & %=
%    \frac{1}{\sqrt{2}}
%    \mqty[\left(
%            x_0 \, \omega^0
%            +
%            x_1 \, \omega^0
%        \right)  \\ 
%        \left(
%            x_0 \, \omega^0
%            +
%            x_1 \, \omega^1
%        \right) 
%    ]
%    & &  =
%    \frac{1}{\sqrt{2}} \mqty[3 \\ -1] \\[1em]}
%    \onslide<3->{
%    & =
%    \underbrace{\frac{1}{\sqrt{2}}
%    \mqty[\omega^0 & \omega^0 \\ \omega^0 & \omega^1]}_{\text{DTF %matrix}}
%    \mqty[x_0 \\ x_1]
%    & &  =
%    \frac{1}{\sqrt{2}} \mqty[3 \\ -1] \\[1em]}
%\end{align*}
%
%\onslide<4>{Complexity: \(O(N^2)\). However, FFT is \(O(N \log %N)\).}
%\end{frame}
%
%\begin{frame}{QFT}
%\begin{align*}
%    \ket{\psi} = \mqty[\alpha_0 \\ \alpha_1 \\ \vdots \\ %\alpha_{N-1}] 
%    & \xrightarrow{\text{QFT}}
%    \mqty[\beta_0 \\ \beta_1 \\ \vdots \\ \beta_{N-1}] 
%    \qquad 
%    \beta_k = \frac{1}{\sqrt{N}} \sum_{j=0}^{N-1} \alpha_j \; %\exp(\frac{2\pi i \cdot j \cdot k}{N})
%\end{align*}
%
%QFT matrix is still \(N \times N\) but its circuit is \(O((\log %N)^2)\) long. 
%
%\alert{Exponential speedup}. 
%\end{frame}
