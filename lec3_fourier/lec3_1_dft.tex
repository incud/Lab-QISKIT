
\begin{frame}{DFT}
\onslide<1->{
\[
    \mqty[x_0 \\ x_1 \\ \vdots \\ x_{N-1}] 
    \xrightarrow{\text{DFT}}
    \mqty[y_0 \\ y_1 \\ \vdots \\ y_{N-1}] 
    \qquad 
    y_k = \frac{1}{\sqrt{N}} \sum_{j=0}^{N-1} x_j \; \exp(\frac{2\pi i}{N} \cdot j \cdot k)
\]}
\onslide<2->{
\[
    \mqty[y_0 \\ y_1 \\ \vdots \\ y_{N-1}] 
    \xrightarrow{\text{DFT}^{-1}}
    \mqty[x_0 \\ x_1 \\ \vdots \\ x_{N-1}] 
    \qquad 
    x_j = \frac{1}{\sqrt{N}} \sum_{k=0}^{N-1} y_k \; \exp(-\frac{2\pi i}{N} \cdot j \cdot k)
\]}
\end{frame}

% ===========================================

\begin{frame}{DFT}

\[
    \mqty[x_0 \\ x_1 \\ \vdots \\ x_{N-1}] 
    \xrightarrow{\text{DFT}}
    \mqty[y_0 \\ y_1 \\ \vdots \\ y_{N-1}] 
    \qquad 
    y_k = \frac{1}{\sqrt{N}} \sum_{j=0}^{N-1} x_j \; \alert{\omega}^{j \cdot k}
\]
\[
    \mqty[y_0 \\ y_1 \\ \vdots \\ y_{N-1}] 
    \xrightarrow{\text{DFT}^{-1}}
    \mqty[x_0 \\ x_1 \\ \vdots \\ x_{N-1}] 
    \qquad 
    x_j = \frac{1}{\sqrt{N}} \sum_{k=0}^{N-1} y_k \; \alert{\omega}^{-j \cdot k}
\]

\bigskip
\[ \alert{\omega} = \text{nth root of unity} \qquad \omega^{N+i}=\omega^i \]
\end{frame}

% ===========================================

\begin{frame}{DFT example}
Calculate \( \mqty[y_0 \\ y_1] = \mathrm{DFT}\left(\mqty[x_0 \\ x_1]\right) \) where \( \mqty[x_0 \\ x_1] = \mqty[1 \\ 2] \).

\alert{\medskip Note: \(\omega^0 = 1\), \(\omega^1 = -1\)}. 

\onslide<2->{
\begin{align*}
    y_0
    & = \frac{1}{\sqrt{2}}\left(
        x_0 \, \omega^{0 \cdot 0}
        + x_1 \, \omega^{0 \cdot 1}
    \right)
    & & = \frac{1}{\sqrt{2}}\left(
        1 \cdot 1
        + 2 \cdot 1
    \right)
    & & = \frac{1}{\sqrt{2}} (3) \\
    y_1
    & = \frac{1}{\sqrt{2}}\left(
        x_0 \, \omega^{1 \cdot 0}
        + x_1 \, \omega^{1 \cdot 1}
    \right) 
    & & = \frac{1}{\sqrt{2}}\left(
        1 \cdot 1
        + 2 \cdot (-1)
    \right)
    & & = \frac{1}{\sqrt{2}} (-1) \\
\end{align*}}
\end{frame}

% ===========================================

\begin{frame}{\(\text{DFT}^{-1}\) example}
Calculate \(\mqty[x_0 \\ x_1] = \mathrm{DFT}^{-1}\left(\mqty[y_0 \\ y_1]\right)\).

\alert{\medskip Note: \(\omega^0 = 1\), \(\omega^1 = -1\)}. 

\onslide<2->{
\begin{align*}
    x_0
    & = \frac{1}{\sqrt{2}}\left(
        y_0 \, \omega^{-0 \cdot 0}
        + y_1 \, \omega^{-0 \cdot 1}
    \right)
    & & = \frac{1}{\sqrt{2}}\left(
        \frac{3}{\sqrt{2}}
        + \frac{-1}{\sqrt{2}}
    \right)
    & & = 1 \\
    x_1 
    & = \frac{1}{\sqrt{2}}\left(
        y_0 \, \omega^{-1 \cdot 0}
        + y_1 \, \omega^{-1 \cdot 1}
    \right)
    & & = \frac{1}{\sqrt{2}}\left(
        \frac{3}{\sqrt{2}}
        - \frac{-1}{\sqrt{2}}
    \right)
    & & = 2 \\
\end{align*}}
\end{frame}

% ===========================================

\begin{frame}[fragile]{DFT Matrix}
\[ \mathrm{DFT}\left(\mqty[x_0 \\ x_1]\right) 
= \underbrace{\frac{1}{\sqrt{2}}
\mqty[\omega^0 & \omega^0 \\ \omega^0 & \omega^1]}_{\text{DTF matrix}}
\mqty[x_0 \\ x_1] 
\]

DTF is \(N\times N\) matrix multiplication. \bigskip

\begin{minted}{python}
from scipy.linalg import dft
import numpy as np
np.set_printoptions(precision=4, suppress=True)

dft(2)
# array([[ 1.+0.j,  1.+0.j],
#        [ 1.+0.j, -1.-0.j]])
\end{minted}
\end{frame}

% ===========================================

\begin{frame}{Note on \code{dft} method}
\begin{itemize}
    \item<1-> Use scipy's \code{dft} to check the math, but \alert{check its specification}
    \item<2-> you need to add \(1/\sqrt{N}\) factor
    \item<3-> primitive n-th root is \(\omega=\exp(\alert{-}\frac{2\pi}{N})\)
    \begin{itemize}
        \item<4-> such matrix is the complex conjugate of QFT's matrix
        \item<5-> both are correct results
    \end{itemize}
\end{itemize}
\end{frame}

% ===========================================

\begin{frame}[fragile]{Note on \code{dft} method}
\begin{itemize}
    \item Use scipy's \code{dft} to check the math, but \alert{check its specification}
    \item you need to add \(1/\sqrt{N}\) factor
    \item primitive n-th root is \(\omega=\exp(\alert{-}\frac{2\pi}{N})\)
    \begin{itemize}
        \item such matrix is the complex conjugate of QFT's matrix
        \item both are correct results
    \end{itemize}
\end{itemize}

\bigskip
\begin{minted}{python}
import numpy as np
dft_matrix = (1/np.sqrt(N)) * dft(N).conj()

# array([[ 0.5+0.j ,  0.5+0.j ,  0.5+0.j ,  0.5+0.j ],
#        [ 0.5+0.j ,  0. +0.5j, -0.5+0.j , -0. -0.5j],
#        [ 0.5+0.j , -0.5+0.j ,  0.5-0.j , -0.5+0.j ],
#        [ 0.5+0.j , -0. -0.5j, -0.5+0.j ,  0. +0.5j]])
\end{minted}
\end{frame}
