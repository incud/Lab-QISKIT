\begin{frame}{Product form of QFT}
Set \(N=2^n, n=3\) and \(\ket{j}\) element of the computational basis.
\[ \mathrm{QFT}: \ket{j} \to \frac{1}{\sqrt{2^3}} \sum_{j=0}^{2^3-1} \omega^{jk} \ket{k} \]

Write \(\ket{j}\) in its binary form \(\ket{j_1 j_2 j_3}\). The above formula is equivalent to
\begin{align*}
    \mathrm{QFT}: \ket{j_1 j_2 j_3} \to 
    \frac{1}{\sqrt{2^3}} &
    (\ket{0} + e^{2\pi i 0.j_3} \ket{1}) \, \otimes \\
    & (\ket{0} + e^{2\pi i 0.j_2j_3} \ket{1}) \, \otimes \\
    & (\ket{0} + e^{2\pi i 0.j_1j_2j_3} \ket{1})
\end{align*}
(See Nielsen\&Chuang p.218)
\end{frame}

\begin{frame}{\(CR_k\) gate}
\[ CR_k = \mqty[
    1 & 0 & 0 & 0 \\ 
    0 & 1 & 0 & 0 \\ 
    0 & 0 & 1 & 0 \\ 
    0 & 0 & 0 & e^{2\pi i/2^k}] 
\]
\[ CP(\theta) = \mqty[
    1 & 0 & 0 & 0 \\ 
    0 & 1 & 0 & 0 \\ 
    0 & 0 & 1 & 0 \\ 
    0 & 0 & 0 & e^{i \theta}] 
\]

\[ CR_k = CP(2 \pi/2^k) \]
\end{frame}

\begin{frame}{QFT 3-qubit}
\begin{center}
\begin{quantikz}[]
    \lstick{\(\ket{j_0}\)}\qw & \qw      & \ctrl{2}   & \qw        
                              & \qw      & \ctrl{1}   & \gate{H}   
                              & \swap{2} & \qw \\
    \lstick{\(\ket{j_1}\)}\qw & \qw      & \qw        & \ctrl{1}        
                              & \gate{H} & \gate{R_2} & \qw 
                              & \swap{1} & \qw \\
    \lstick{\(\ket{j_2}\)}\qw & \gate{H} & \gate{R_3} & \gate{R_2}  
                              & \qw      & \qw        & \qw
                              & \targX{} & \qw \\
\end{quantikz}
\end{center}
\end{frame}

\begin{frame}[fragile]{QFT 3-qubit (1)}
\begin{minted}{python}
import numpy as np
from math import pi
from scipy.linalg import dft
from qiskit import * 
from qiskit.circuit.library import QFT

N, n = 8, 3
x = np.array([1, 2, 3, 4, 5, 6, 7, 8])
while np.linalg.norm(x) != 1: x = x / np.linalg.norm(x)

# classical procedure
y_dft = (1/np.sqrt(N)) * dft(N).conj().dot(x)
\end{minted}
\end{frame}


\begin{frame}[fragile]{QFT 3-qubit (2)}
\begin{minted}{python}
def qft_rotations(circuit, n):
    if n == 0: return circuit
    n -= 1
    circuit.h(n)
    for qubit in range(n):
        circuit.cp(pi/2**(n-qubit), qubit, n)
    qft_rotations(circuit, n)
    
def swap_registers(circuit, n):
    for qubit in range(n//2):
        circuit.swap(qubit, n-qubit-1)
    return circuit
\end{minted}
\end{frame}

\begin{frame}[fragile]{QFT 3-qubit (3)}
\begin{minted}{python}
the_qc = QuantumCircuit(n)
the_qc.initialize(x, range(0, n))
qft_rotations(the_qc, n)
swap_registers(the_qc, n)
sv_sim = Aer.get_backend('statevector_simulator')
y_qft = execute(the_qc, sv_sim).result().get_statevector()
\end{minted}

\smallskip Check the math:
\begin{minted}{python}
np.allclose(y_dft, y_qft)
\end{minted}

\smallskip Equivalent to Qiskit function:
\begin{minted}{python}
the_qc = QuantumCircuit(n)
the_qc.initialize(x, range(0, n))
the_qc.append(QFT(n), range(0, n))
y_qft_qiskit = execute(the_qc, sv_sim).result().get_statevector()
np.allclose(y_qft, y_qft_qiskit) # True
\end{minted}
\end{frame}