\begin{frame}{The Quantum Circuit}
\begin{itemize}
    \item<1-> define the circuit (\code{QuantumCircuit})
    \begin{itemize}
        \item<2-> 1+ \code{QuantumRegister} (w/ optional name)
        \item<3-> 0+ \code{ClassicalRegister} (w/ optional name)
    \end{itemize}
    \item<4-> draw the circuit (method \code{draw} of \code{QuantumCircuit})
    \item<5-> add the gates (methods of \code{QuantumCircuit})
    \begin{itemize}
        \item<6-> \code{.x(qubit)}
        \item<8-> \code{.h(qubit)}
        \item<9-> \code{.cx(control qubit, target qubit)}
        \item<10-> \code{.rx(theta, qubit)}
        \item<11-> \code{.crx(theta, control qubit, target qubit)}
        \item<12-> optionally use indexes instead of qubit
        \begin{itemize}
            \item<13-> depends on the order you insert the registers
            \item<14-> be consistent w/ the notation
        \end{itemize}
    \end{itemize}
    \item<15-> add the measurements (method of \code{QuantumCircuit})
    \begin{itemize}
        \item<15-> \code{.measure(qubit, cbit)}
    \end{itemize}
\end{itemize}
\end{frame}


\begin{frame}[fragile]{The code (1)}
\begin{minted}{python}
from qiskit import *

# first circuit
qr = QuantumRegister(2, 'qreg')
qc = QuantumCircuit(qr)
qc.draw()

qc.h(qr[0])
qc.cx(qr[0], qr[1])
qc.rx(0.1, qr[1])
qc.draw()
\end{minted}
\end{frame}


\begin{frame}[fragile]{The code (2)}
\begin{minted}{python}
# second circuit
qra = QuantumRegister(2, 'qa')
qrb = QuantumRegister(2, 'qb')
cr = ClassicalRegister(2, 'creg')
qc = QuantumCircuit(qra, qrb, cr)
qc.cx(qra[0], qra[1])
qc.cy(qrb[0], qrb[1])
qc.cx(0, 1)
qc.cy(2, 3)
qc.measure(0, 0)
qc.draw()
\end{minted}
\end{frame}